\documentclass[twoside,11pt]{homework}

\coursename{Unsupervised Learning (2018 Fall)} 

\studname{Youki Cao}    % YOUR NAME GOES HERE
\studmail{}% YOUR UNI GOES HERE
\hwNo{1}                   % THE HOMEWORK NUMBER GOES HERE
\date{Oct 5 2018} % DATE GOES HERE

% Uncomment the next line if you want to use \includegraphics.
%\usepackage{graphicx}

\begin{document}
\maketitle

\section*{Problem 1}

\subsection*{- "k-means++: The Advantages of Careful Seeding" by Arthur and Vassilvitskii}

The paper proposes a great improvement for the initializing part in the k-means algorithm. The classical k-means method offers no accuracy guarantees, but k-means++ is O(log $k$)-competitive with optimal clustering. Instead of choosing the centers at random, k-means++ weighs the data points according to their squared distance from the closest center that has already been chosen. The significance of it lies in the enhancement of both speed and accuracy.
\newline
Then I'll show some technical analysis of the paper. The main idea is to prove the expectation of the potential function $\phi$ is $O(\log k)$-competitive. Firstly, we need to prove theorem 3.1: If $C$ is constructed with k-means++, then eht corresponding potential function $\phi$ satisfies, $E[\phi] \leq 8(\ln k+2)\phi_{OPT}$. To prove it, we need to clarify two lemmas: the first is Lemma 3.2, let $A$ be a cluster from $C_{OPT}$, let $C$ be just one cluster uniformly at random from $A$, then $E[\phi(A)] \leq 2\phi_{OPT}(A)$. To prove it, we need to use a small conclusion $E[||x-y||^2] = 2E[||x-E(x)||^2]$ . Lemma 3.2 ensures us the first step of k-means++. For remaining centers, we need to use lemma 3.3: let $A$ be an arbitrary cluster in $C_{OPT}$, and let $C$ be an arbitrary clustering. If we add a random center to $C$ from A, chosen with $D^2$ weighting, then $E[\phi(A)] \leq 8\phi_{OPT}(A)$. So far we know that as long as it chooses centers from each cluster of $C_{OPT}$, then the algorithm is competitive. Finally we use lemma 3.4 to obtain the desired bound $E[\phi]$. And then we can finally prove the theorem.
\newline
The next part is to show that k-means++ is no better than $\Omega(\log k)$ competitive. Fix $k$, and then choose $n, \Delta, \delta$, construct $\mathcal(X)$ with $n$ points. First choose ${c_1, \cdots,c_k}$ so that $||c_i-c_j||^2=\Delta ^2 - (\frac{n-k}{n})\delta ^2$. And now add data points centered at $c_i$, and each distance $\sqrt{\frac{n-k}{2n}}\delta$ from $c_i$. And then we prove our algorithm is $\Omega(\log k)$ worse than the optimal clustering in this case. Now we need to prove lemma 4.1 to give a lower bound of $E[\phi']$ by induction. And therefore we get this kind of seeding is no better than $2(\ln k)$-competitive, which shows the analysis is tight.
\newline
Apart from the main results shown above, there is also some corollaries in this paper. And there is a general theorem shows if $C$ is constructed with $D^l$ seeding, then the corresponding potential function $\phi^{[l]}$ satisfies, $E[\phi^{[l]}] \leq 2^{l+2}(\ln k+2)\phi_{OPT}$.
In a nutshell, this kind of seeding method performs better in theory and experiments.





\subsection*{- "Hartigan’s Method: k-means Clustering without Voronoi" by Telgarsky and Vattani}

From this paper, three formulations of the Hartigan's Method's heuristics were given (including the original version). From these formulations, we can get some great properties of Hartigan's method, and we also get the data partition is always quite separated from the induced Voronoi partition. Based on the theoretical analysis and empirical tests, it concludes that Hartigan's method has good optimization performance and good running time, compared to Lloyd's method.\newline
Then I'll introduce some technical results with more details. The main idea of Hartigan's Method is repeatedly pick a point, and determine its optimal cluster assignment by comparing the value of a particular function before and after moving that point to another cluster. Here we have the bias-variance decomposition of k-means cost: $\phi(C,z)=\phi(C)+C||\mu (C)-z||^2$. And denote the cost of merging two clusters $A,B$ by $\Delta(A,B)=\phi(A\cup B)-\phi(A)-\phi(B)=\frac{AB}{A+B}||\mu(A)-\mu(B)||^2$. These two formulas will be used in the following context. To ensure we get to the optima, we need to set a terminal condition.
\begin{enumerate}
    \item 
    Holistic Formulation\newline
    This is the most natural form. After we move $x$ from its previous cluster $S$ to another cluster $T$, we have the delta value
    $$\phi(x;S,T)=\phi(S)+\phi(T)-\phi(S\backslash\{x\})-\phi(T\cup \{x\})$$
    Using the formula in the last paragraph, we have the first termination whether$\Phi(x;S,T)>0$, where
    $$\Phi(x;S,T)=\frac{T}{T+1}||\mu(T)-x||^2-\frac{S}{S-1}||\mu(S)-x||^2$$
    Here the most crucial part is how to select a point that could improve most.
    \item
    Point-relative Formulation\newline
    From the termination condition above , we yields
    $$||\mu(C_j)-x_i||<||\mu(C_{y_i})-x_i||\sqrt{\frac{C_{y_i}(C_j+1)}{C_j(C_{y_j}-1)}}$$
    So here we can give the second formulation. We define $\alpha(S,T)=\sqrt{S(T+1)/(T(S-1))}$, and we will implement the iteration if $\exists i,j$, s.t. $||\mu(C_j)-x_i|| < \alpha (C_y,C_j)||\mu (C_{y_i})-x_i||$. From here it is not hard to prove this method has some good properties: (1) The cost sequence is strictly decreasing (Lloyd has this property too); (2) The resulting partition has no empty clusters; (3) The resulting partition has distinct means. The property(2) and (3) are not satisfied by Lloyd.
    \item
    Cluster-relative Formulation\newline
    The termination condition is there exists no $i,j$ so that $||x_i-v(C_y,C_j)|| > \rho(C_y, C_j)$, where $\rho(S;T)=\alpha ||v(S;T)-\mu(S)||$. This is simply the Voronoi partition. And we can deduce the following theorem. Let $S,T$ be two clusters as provided by the termination of $H_3$. Then every point of $S\cup T$ is at a distance of at least $||\mu(S) - \mu(T)||/(2S+2T)$ from the Voronoi boundary (hyperplane) between $S$ and $T$. We have the k-means global optimum must be a (local) optimum of Hartigan's method.\newline\newline
    Therefore we have the Hartigan's method has opportunities to escape the local optima of Lloyd's method. And we can show that Hartigan's method can outperform Lloyd's method by quantifying the gap between the circlonoi partition and the Voronoi partition.
\end{enumerate}






\section*{Problem 2}

For any instance of Vertex-Cover* $G = (V,E)$, Let $W = \{w_e, e \in E\}$, which is a new set of vertices, each of them is an edge in Vertex-Cover*. Define the distance over $G \cup W$ as below:
\begin{itemize}
    \item
    $\rho(u,v) = 1$ if $u,v \in V$ with $(u,v) \in E$;
    \item
    $\rho(u,v)=1$ if $u \in V$ and $v = (u,t) \in E$ for some $t$;
    \item
    $\rho (u,v)=2$ otherwise.
\end{itemize}

\begin{lemma}
The solution of the Vertex-Cover* problem $G$ is $V'$, with size $|V'| \leq k$ iff $(V \cup W, \rho)$ has a k-center solution of 1.
\end{lemma}

\begin{proof}
$[\Rightarrow]$ There exists a $V' \subset V$ such that $V' \cup (\cup_{v' \in V'} \cup_{e(v',v) \in E} v)=V$, and $|V'|=k.$ Consider any $v \in V \cup W$:
\begin{itemize}
    \item 
    If $v \in V'$, then $\rho (v, V') =0$;
    \item
    If $v \in V-V'$, then there exists some edge $e(u,v) \in E$, and $u \in V'$. By definition we have $\rho(v,V') = \rho(v,u)=1$; 
    \item
    If $v \in W$, denote $v=(p,q) \in E$. Then either $p$ or $q$ is in $V'$. So $\rho (v, V') \leq \min \{\rho(v,p), \rho(v,q)\}=1$.
\end{itemize}
So we have $\rho(v, V') \leq 1$ for all $v \in V \cup W$. On the other hand, all the distance is at least 1. Therefore $(V \cup W, \rho)$ has a k-center solution of 1.
\newline
\newline
$[\Leftarrow]$ Let $T$ be a k-center solution of cost 1. If $\exists w_e \in W \cap T$, where $w_e = (u,v)$. Then replace $T$ with $(T - \{w_e\})\cup \{u\}$. Here we get a new set $T' \subset V$. $|T'| \leq k$ and $\rho (v, T') \leq 1$ for all $v \in V \cup W$. Then I'll prove $T'$ is a vertex-cover of $V$. For every $w_e = (u,v) \in E$, since $\rho(w_e, T')=1$, then either $u \in T'$ or $v \in T'$. So $T'$ is a vertex-cover of $V$, with size $\leq k$.
\end{proof}
By Lemma we know each instance in Vertex-Cover* can be modified in polynomial time into an instance of k-center problem, and vice versa.
Using the fact that Vertex-Cover* is NP-hard, we know that k-center problem is NP-hard. \#
\newline




\section*{Problem 3}
Balanced k-means problem is an NP-hard problem.\newline
We just need to prove balanced 2-means problem is an NP-hard problem, and then we can naturally know that balanced k-means problem is an NP-hard problem.
Here we also use NAE-3SAT* to prove the hardness. Introduce the same distance matrix $D(\phi)$ as def25. From lemma26 we know $\phi$ is NAE-3SAT* satisfied iff $D(\phi)$ admits to a generalized 2-means cost of cost$(\phi)=n-1+2\delta m/n$. Also from lemma 27, we know that if any partition contains an element and its negation, then the cost will be greater than $n-1+2\delta m/n$. That means if the partition is not balanced then, the cost will be greater than $n-1+2\delta m/n$. So here we actually loosen lemma 27, and add an condition that partition is balanced. It's easy to get then. After that, we can use the same embedding method to prove balanced 2-means is NP-hard. \#



\section*{Problem 4}
\begin{enumerate}
    \item 
    I'll construct a $Y \subset X$ that is both an $\epsilon$-cover and an $\epsilon$-packing.
    \begin{enumerate}
        \item 
        Select $x_1$ arbitrarily from $X$. Consider the closed ball $B_1 :=B[x_1, \epsilon]=\{x: \rho (x_1, x) \leq \epsilon\}$.
        \item
        Select $x_i$ $(i \geq 2)$ arbitrarily from $X - \cup_{j=1}^{i-1} B_j$. Denote $B_i$ as the closed ball  $B_j :=B[x_j, \epsilon]=\{x: \rho (x_j, x) \leq \epsilon\}$.
        \item
        Repeat step(b) until $\cup_{j}B_j = X$. Let $Y = \cup_{j}\{x_j\}$, which is both an $\epsilon$-cover and an $\epsilon$-packing.
    \end{enumerate}
    Then I'll prove the correctness.
    \begin{proof}
      Firstly, I'll show that this algorithm will stop within a finite number of times. Because $X$ is a compact metric space, it is totally bounded. So there exists a finite collection of open balls in $X$ of radius $\epsilon/4$ whose union contains $X$. Therefore there are finite disjoint open balls $M = \{O_1, \cdots, O_m\}$ of radius $\epsilon/4$ so that their union will cover $Y \subset X$. They are disjoint because the distance of each two points in $Y$ ie greater than $\epsilon$. So $|Y| \leq |M| < \infty$. Therefore this algorithm will stop within finite times.
      
      Secondly, I'll prove that $Y$ is an $\epsilon$-cover. Because $\cup_j B_j = X$, $\forall x \in X$, $\exists i$ s.t. $x \in B_i$. So we have $x_i \in Y$ and $\rho(x, x_i) \leq \epsilon$.
      
      Lastly, I'll show that $Y$ is an  $\epsilon$-packing. $\forall i<j$, we have $x_j \notin  \cup_{j=1}^{i-1} B_j$, that shows $x_j \notin B_i$. That is $\rho(x_i, x_j) > \epsilon$ for $\forall i \neq j$.
    \end{proof}
    
    
    \item
    
      For the first part of the inequality, I'll prove by contradiction. Assume that there exists an $\epsilon$-packing $P=\{p_1,\cdots,p_M\}$, and an $\epsilon/2$-cover $C = \{c_1, \cdots,c_N \}$, and $M \geq N+1$. Then from Pigeonhole principle, there exist $p_i$ and $p_j$ ($i \neq j$) that belong to the same closed ball $B[c_k, \epsilon]$. So $\rho(c_k, p_i) \leq \epsilon/2$ and $\rho(c_k, p_j) \leq \epsilon/2$. From the triangle inequality of metric space, we have
      $$\rho(p_i,p_j)\leq \rho(p_i,c_k)+\rho(p_j,c_k) \leq \epsilon$$
      This leads to a contradiction because $\rho(p_i,p_j)>\epsilon$ for an $\epsilon$-packing. Hence the size of any $\epsilon$-packing is no larger than the size of any $\epsilon/2$-cover. So $P_\epsilon (X) \leq N_{\epsilon/2} (X)$.
      \newline
      \newline
      For the second part of the inequality, denote $P =\{p_1,\cdots,p_M\}$ is a max $\epsilon/2$-packing. Then for any $x \in X - P$, $\exists 1 \leq i \leq M $, so that $\rho(x_i, x) \leq \epsilon/2$. (Otherwise, $P\cup \{x\}$ is a $\epsilon/2$-packing, whose size is larger than the size of $P$. This will lead to a contradiction of our maximal assumption.) So we have $x \in B(x_i, \epsilon / 2)$. Therefore $P$ is a $\epsilon /2$-cover. So the size of the smallest $\epsilon/2$-cover is no larger than the size of $p_M$. So $N_{\epsilon/2}(X) \leq P_{\epsilon/2}(X)$. 
      \newline\newline
      Therefore we conclude that $P_\epsilon(X) \leq N_{\epsilon/2} (X) \leq P_{\epsilon/2}(x)$\#
        
        
    \newpage
    \item
    First we estimate the lower bound of $N_\epsilon (B^d(0,1))$.\newline
    Consider an $\epsilon$-cover of $B^d(0,1)$ of size $N=N_\epsilon (B^d(0,1))$: $C = \{c_1,\cdots,c_N\}$. Then we have $B^d(0,1) \subset \bigcup_{i=1}^N c_i$, so
    $$\text{vol}(B^d(0,1)) \leq N \text{vol}(B^d(0,\epsilon)) = N\epsilon^d \text{vol}(B^d(0,1))$$
    So $N \geq 1/\epsilon^d$.
    \newline
    \newline
    Next let's estimate the upper bound.\newline
    For any $\epsilon$-packing of $B^d(0,1)$ of size $M=P_\epsilon (B^d(0,1))$: $P=\{p_1,\cdots,p_M\}$. Then we consider the balls $B^d (p_i, \epsilon/2)$. These balls are disjoint because $P$ is an $\epsilon$-packing. Besides, all balls lie in $B^d (0, 1+\epsilon/2)$. Therefore, $\bigcup_{j=1}^M B^d(p_i, \epsilon/2) \subset B^d(0, 1+\epsilon/2)$.
    $$M \text{vol} (B^d(p_i, \epsilon/2)) \leq \text{vol} (B^d(0,1+\epsilon/2))$$
    $$M(\frac{\epsilon}{2})^d \text{vol}(B^d(0,1)) \leq (1+\epsilon/2)^d \text{vol}(B^d(0,1))$$
    Hence we have $M \leq (\frac{1+\epsilon/2}{\epsilon/2})^d = (\frac{\epsilon+2}{\epsilon})^d$. So $N_\epsilon (B^d(0,1)) \leq M \leq (\frac{\epsilon+2}{\epsilon})^d$.
    \newline
    \newline
    In conclusion, we have $(\frac{1}{\epsilon})^d \leq N_\epsilon (B^d(0,1))  \leq (\frac{\epsilon+2}{\epsilon})^d$. \#
    
    \item
    Consider the singular value decomposition of $A = U\Sigma V^*$, $\Sigma$ is a $m \times n$ matrix. $\Sigma = (\text{diag}(\lambda_1,\cdots,\lambda_m))$. Here $\{\lambda_i\}$ are the singular values of $A$, in a decreasing order of their absolute value. 
    $$\sigma_{\max} (A) = \underset{ x \in \mathcal{R}^n, ||x||=1}{\max}||Ax|| = \underset{ x \in \mathcal{R}^n, ||x||=1}{\max}||U \Sigma V^* x|| = \underset{ x \in \mathcal{R}^n, ||x||=1}{\max}||\Sigma x|| = |\lambda_1|$$
    Then for any $\epsilon$-cover $C_0$ of $S^{n-1}$, we have $V^* C_0$ is also an $\epsilon$-cover of $S^{n-1}$. So we need to find the difference between $|\lambda_1|$ and $ \underset{ x \in C}{\max}||U\Sigma x|| = \underset{ x \in C}{\max}||\Sigma x||$. Let $x_0 =  \underset{x \in S^{n-1}}{\arg\min} ||\sigma x|| = (1, 0,\cdots,0)$. Because $C$ is an $\epsilon$-cover, there exists some $x = (x_1, \cdots,x_n) \in C$ so that $||x-x_0||_2^2 \leq \epsilon ^2$. So we have
    $$(1-x_1)^2 + \sum_{i=2}^n x_i^2 \leq \epsilon ^2$$
    $$\sum_{i=1}^n {x_i}^2=1$$
    So we get $x_1 \geq 1-\epsilon^2/2$.
    Therefore $||\Sigma x||_2^2 = \sum_{i=1}^m \lambda_i^2 x_i^2 \geq |\lambda_1|^2(1-\epsilon^2/2)^2$.\newline
    So $\sigma_{\max} \leq \frac{\sigma_C}{1-\epsilon^2/2}$.
    Therefore the tight upperbound of $\sigma_{\max}$ is $\frac{\sigma_C}{1-\epsilon^2/2}$. If $m<n$, it can be achieved when $x_1=1-\epsilon^2/2$, $x_{m+1} = \sqrt{1-(1-{x_1}^2)}$,
    $x_i=0$ otherwise. If $m \geq n$, then it cannot be achieved, so it is just a loose tight upperbound.\newline
    Plus it is natural that $\sigma_C \leq \sigma_{\max}$, which is a tighter lower-bound of $\sigma_{\max}$. \#
    
    
    
    
    
\end{enumerate}


\section*{Problem 5}
\begin{enumerate}
    \item 
    Let $|V|=n$, $|E|=m$. $d_S$ is the sub-matrix only has rows of edges in S, so its dimension is $|S|\times n$. Let $e =(i,j) \in S$ is an edge. Then the $e^{th}$ row of $d_S$ is all zero expects that $i_{th}$ and $j_{th}$ columns. There one of the entries is 1, and the other is -1.
    So $d_S$ has differential operator equal to the graph $G_S=(V,S)$, that is the graph eliminating the edges in $E\backslash S$ from the original graph $G$.
    
    \item
    I'll show each entries of L is correct for $L=\sum_{i=1}^k L_i$ by two cases: entries in diagonal, and others. Denote $L = (l_{pq})_{n\times n}$, $L_i = (l_{i_{pq}})_{n\times n}$. By definition we have $L=D-A$, where $D$ is the degree matrix and $A$ is the adjacency matrix. So for $1 \leq p,q \leq n$, we have
    \begin{equation}
        l_{pq} = 
        \begin{cases}
        \text{deg}(v_p)& p=q\\
        -1& p \neq q \text{ and} (v_p,v_q) \in E\\
        0& o.w.
        \end{cases}
    \end{equation}
    \begin{enumerate}
        \item 
        For entries $l_{pp}$ in diagonal, $l_{pp} = \text{deg}(v_p)$, that is the number of edges that connect $v_p$. Because $E_1,\cdots,E_k$ is a partition of $E$, $E = \bigcup_{i=1}^k E_k$ is a disjoint union of all the edges. So each edge connecting $v_p$ will exist and only exist in some $E_i$. The number of edges connect $v_p$ is equal to the sum of the number of edges connect $v_p$ in each $E_i$, that is $l_{i_{pp}}$. So $\sum_{i=1}^k l_{i_{pp}}=\sum_{i=1}^k \text{deg}_i (v_p)  = \# \text{edges connect }v_p = \text{deg}(v_p)=l_{pp}$.
        \item
        For entries $l_{pq}$ with $p\neq q$. If there is no edge of $(v_p, v_q)$, then $l_{pq}=0$. On the other hand, there is naturally no edge of $(v_p, v_q)$ in any $E_i$. (Because $E_i$ is a subset of $E$.) Therefore $\forall 1 \leq i \leq n$, $l_{i_{pq}}=0$. So we have $l_{pq}=0=\sum_{i=1}^n l_{i_{pq}}$ in this situation.\newline
        \newline
        Besides, when there exists an edge $e=(v_p,v_q) \in E$, then we have  $l_{pq}=-1$ by definition. Because $\{E_i\}_{i=1}^k$ is a disjoint union of all edges, there will exist and only exist one $t$ so that $e=(v_p,v_q) \in E_t$. So we have $l_{t_{pq}}=-1$, and $l_{i_{pq}}=0 \text{ for } i\neq t$. Hence $\sum_{i=1}^k l_{t_{pq}}=-1+0+\cdots+0=-1= l_{pq}$.
        \newline\newline
        Therefore we conclude that $L=\sum_{i=1}^k L_i$. \#
    \end{enumerate}
    
    \item
    The graph $\mathcal{G}=(V\cup V',E\cup E')$ is the disjoint union of $G$ and $G'$. Consider $G \cup v(G') := (V_G \cup V_{G'}, E_G)$ and $G' \cup v(G) := (V_G \cup V_{G'}, E_{G'})$. Then we have,
    $$L_{G \cup v(G')}=
    \left[
    \begin{matrix}
    L_G &0\\
    0&0
    \end{matrix}
    \right]$$
    $$L_{G' \cup v(G)}=
    \left[
    \begin{matrix}
    0 &0\\
    0&L_{G'}
    \end{matrix}
    \right]$$
    This is natural because we only add isolated vertex to the graph, without doing anything on edges. Then we have $\mathcal{G}=(V\cup V',E\cup E') = (L_{G \cup v(G')})\cup (L_{G' \cup v(G)})$
    
    Here $E$ and $E'$ are the disjoint, from question 5.2 we have:
    $$L_\mathcal{G} =L_{L_{G \cup v(G')}} + L_{L_{G' \cup v(G)}} = 
    \left[
    \begin{matrix}
    L_G &0\\
    0&L_{G'}
    \end{matrix}
    \right]$$
    Here the first $|V|$ column/row represent vertices in $G$, and the rest $|V'|$ column/row represent vertices in $G'$.
    \item
    For the $p_{th}$ row of L, consider each entries $l_{pq}$.
    If $p =q$, $l_{pq}=\text{deg}(v_p).$
    If $p \neq q$, $l_{pq} =-1$ when $(v_p,v_q)\in E$, else 0. So $\sum_{1 \leq q \leq n, q \neq p} l_{pq} = (-1) * \# \text{ edges connect }v_p = -\text{deg}(v_p)$. Hence $\sum_{q=1}^n l_{pq} = \text{deg}(v_p)-\text{deg}(v_p)=0$. 
    Denote $a=(1,1,\cdots,1)^T$.
    For each row $L_p$, we have $L_p a=\sum_{q=1}^n l_{pq}=0$. So $La=0$.\newline\newline
    Here we got the vector $a \in \ker(L)$. So dim($\ker (L)$)$\geq 1$. By Rank nullity theorem, we have $\text{rank} (L) \leq n-1$. Next I'll show it's a tight upper bound.\newline\newline
    Construct a graph $G=(V,E)$, $V = \{v_1, \cdots,v_n\}$, $E = \{(x_1,x_j)|2 \leq j \leq n\}$. There are $n$ edges and $n$ vertices in total. It's Laplacian is:
    $$L=\left[
    \begin{matrix}
    n-1 & -1 & -1 & \cdots & -1\\
    -1& 1  & 0  & \cdots  &0\\
    -1&0&1&\cdots &0\\
    \cdots\\
    -1&0&0&\cdots &1
    \end{matrix}
    \right]$$
    Assume $x = (x_1, \cdots, x_n)^T \in \ker (L)$. $Lx=0$. Then we get the following equations:
    $$(n-1) x_1-x_2-\cdots -x_n=0$$
    $$-x_1+x_2=0$$
    $$-x_1+x_3=0$$
    $$\cdots$$
    $$-x_1+x_n=0$$
    
    The solution is $x_1=x_2=\cdots =x_n=c$. So the solution space is spanned by the vector $\textbf{1}$, which means dim($\ker (L)$) = 1. Then by rank nullity theorem, we have rank($L$) = $n-1$, which achieves my bound.
    \newline\newline
    In conclusion, the tight upper bound for rand($L$) is $n-1$. $\ker (L)$ always contains $(c,c,\cdots,c)^T$, where $c$ is a constant in $\mathbb{R}$. \#
    
\end{enumerate}


\section*{Problem 6}
\begin{enumerate}
    \item 
    $1 \Rightarrow 2$
    Let $G = (V,E)$. Now we add the edge $e=(i,j)$ to G, $e \notin E$.
    Because $G$ is connected , there exists a simple path from $i$ to $j$ in G. Suppose $e_1, \cdots,e_n$ are the edge sequence defining this path. Then after the new edge $e$ is added, $e_1,\cdots,e_n,e$ define a path from $i$ to itself, that means it will form a circuit, so we have $G$ is maximally acyclic.
    
    \paragraph{$2 \Rightarrow 1$}Because $G$ is connected and maximally acyclic, $G$ is naturally a tree.
    
    \paragraph{$1 \Rightarrow 3$}We'll use induction to prove this part.
    \newline
    For $|V|=2$, it is natual that $|E|=1, |E| = |V|-1$. 
    \newline
    Suppose we have proved that for a tree $G = (V,E)$ with $k$ nodes, $|E| = |V| - 1 = k-1$. Then given a tree $G$ with $k+1$ nodes, we can first remove a leaf node and the edge connected to it, so that we have a new graph $G' = (V', E')$. $G' $ is also connected and acyclic, so $G'$ is a tree with $k$ nodes. By the assumption of the induction, $|E'| = |V'| - 1 = k-1$. $V'$ is the union of $V$ and the removed edge, so we have $|V| = |V'| +1 = k = |G|-1$.
    \newline
    So by induction, we proved the conclusion.
    
    \paragraph{$3 \Rightarrow 1$}For $G = (V,E)$, $|E| = |V|-1$. Because $G$ is connected graph, we only need to show it is acyclic. Assume that $G$ has a cycle. Then we can remove an edge of that cycle without affecting the connectivity of it. After removing all that edges from all cycles, we obtain a new graph $G' = (V', E')$, which is a spanning tree of $G$. So we have
      $|E| > |E'| = |V'|-1 = |V|-1$
    This contradicts the original claim that $|E| =|V|-1$, so there is no cycle in $G$. That is, $G$ is a tree. \#
    
    \item
    $[\Rightarrow] $
    For $G=(V,E)$, its derivative $d$ is full rank.
    If there exists a circle in the graph, which has $t$ edges and $t$ vertices. Here $t \leq \min(m,n)$. Denote the edge set as $T = \{e_1, e_2, \cdots, e_t\}$. 
    Consider the submatrix of $d$ with all the nodes and only the edges in $T$, denote it as $d_T$, whose dimension is $t \times n$. Because for each nodes in the cycle, it has exactly one "in" and one "out" in this cycle, that is for the column representing these nodes, there is only one "1" and one "-1", whose sum is 0. For other columns, because there is no edges connecting them, the corresponding column in $d_T$ is all zero. Then $\sum_{i=1}^T=0$ because for each column, the sum of the elements is zero. Therefore the rank of $d_T \leq t-1$. So one row of $d$ can be the linear combination of the some other rows. That means, $d$ is not row full rank.
    \newline\newline
    Also for $d$'s each column vectors $\{u_1,\cdots,u_n\}$. Because for each row $(x_1,\cdots,x_n)$, there is exactly one "1" and one "-1". So $\sum_{i=1}^n u_i=\textbf{0}$. That means $d$ is not column full rank too. Therefore, $d$ is not full rank, which causes a contradiction. So we have $G$ has no cycles.
    \newline\newline\newline
    $[\Leftarrow]$ Suppose a graph has no cycles, then it is a tree or some connected components (forest). If it contains several connected components, we can order the nodes and edges according to it, and finally get a block diagonal matrix. In order to show it's full rank we need to prove each block of it is full rank. So it's suffice to show the derivative of a tree $G=(V,E)$ is full rank.
    
    
    Here we can use the conclusion that we'll prove in question 6.3. That is for a tree, after removing the derivative matrix's first column, the remain matrix $\hat{d}$ holds $det(\hat{d}) = \pm 1$. That means $\hat{d}$ is full rank, and so the row vectors of $\hat{d}$ are linearly independent. But if $d$ is not full rank. Because it's a tree, we have $|E|=|V|-1$. That means the row vectors $\{v_i\}_{i=1}^m$ are linearly dependent. So there exists $\{\lambda_i\}_{i=1}^m$ not all equal to zero, satisfying $\sum_{i=1}^m \lambda_i v_i=\textbf{0}$. But here since $\hat{d}$ is removing a column from $d$, we will have for row columns ${\{\hat{v_i}\}}$ of $\hat{d}$, there is $\sum_{i=1}^m \lambda_i \hat{v_i}=\textbf{0}$. This causes a contradiction to linear independence. So we draw the conclusion that $d$ is full rank, which further prove that if $G$ has no cycles, $d$ is full rank.

    
    \item
    
    Firstly I will use induction prove that
    \begin{equation}
        det(\hat{d}) = \pm 1
    \end{equation}
    \begin{enumerate}
        \item Base: Consider a tree with 2 nodes. By question 6(1), there is only one edge. After removing the first column, we have $\hat{d} = (\pm 1)$, which shows the conclusion is right for $|V|=2$.
        \item Inductive Assumption: Assume for any tree $G$ with $n$ nodes, the derivative is $d_{n-1}$. We have $det(\hat{d_{n-1}})=\pm 1$
        \item Inductive Step: Consider a tree $G$ with $n$ nodes, the derivative is $d$. Suppose that the edge $e_j = (1,k)$ connects the first node and the $i_{th}$ node. So after we remove the first column of $d$, the only non-zero entry of $\hat{d}$ is the $k_{th}$ element, which is 1 or -1. So let's compute the determinate of $\hat{d}$ by expanding along the $j_{th}$ row. Denote $\hat{d}'$ as the remain matrix after removing the $j_{th}$ row and $(k-1)_{th}$ column from $\hat{d}$. Then we have 
        $$det(\hat{d}) = \hat{d}_{j,k-1} det(\hat{d}') =\pm det(\hat{d}')$$
        Next, I'll show $\hat{d}'$ is the matrix after removing one column from the derivative of a tree with $n-1 $ nodes. Suppose the graph constructed as below: Remove the first node and the edges connecting it. Then connect all the neighbours of the first node (except the $k_{th}$ node) with the $k_{th}$ node. Here we get a graph with $n-1$ nodes. This graph $G'=(V',E')$ is a tree because $|E'| = |E|-n+(n-1)=|E|-1 = |V|-2 = |V'| -1$.
        
        Then let the $k_{th}$ node to be the first column of the new derivative matrix, and keep the order of other nodes and edges the same as before. Then after removing the first column of the derivative, we will get a matrix exactly same as $\hat{d}'$.
        
        By induction assumption, we have for graph $G'$, $det(\hat{d}')=\pm 1$. 
        
        So we have $det(\hat{d}) = \pm det(\hat{d}') = \pm 1$.
        \item
        Therefore we can conclude that $det(\hat{d}')=1$.
        
    \end{enumerate}
    
    Finally we have
    $$det(\hat{d}^T\hat{d}) = det(\hat{d}^T)det(\hat{d}) =(det(\hat{d}))^{2} = 1$$

    
    
    \item
    For a connected graph $G = (V,E)$, $|V|=n$, $|E| = m$.Denote by $\hat{d}$ the matrix monior of $d$ produced by removing its first column. Let $S \subset E$ be a collection of edges, denote $d_S$ as the submatrix of $d$ where only contains the edges in $S$. Denote the Laplacian matrix of $G$ as $L$, the derivative as $d$. $ d= (a_1, \hat{d})$. From the conclusion of HW0, we have $$L = d^Td = (a_1 \hat{d})^T (a_1 \hat{d}) = \left[
    \begin{matrix}
    a_1^Ta_1 & a_1^T \hat{d} \\
    \hat{d}^T a_1& \hat{d}^T \hat{d} 
    \end{matrix}
    \right]$$
    So we have $L_{1,1} = \hat{d}^T \hat{d}$. And we need to calculate the determinant of $\hat{d}^T \hat{d}$.
    Let $\mathcal{S} = \{S \subset E | |S|=n-1\}$. Then for each $S \in \mathcal{S}$, $\hat{d_S}$ is a $(n-1)\times(n-1)$ matrix. From Cauchy-Binet Formula, we have:
    $$L_{1,1} = det(\hat{d}^T \hat{d}) = \sum_{S \in \mathcal{S}} det(\hat{d_S}^T) det(\hat{d_S})$$
    Let $G_S = (V, S)$. Then we will have the following scenarios.
    \begin{enumerate}
        \item 
        If $G_S$ is a tree. From question 6.3, we have $det(\hat{d_S})=\pm 1$. Therefore we have $det(\hat{d_S}^T) det(\hat{d_S})=1$.
        \item
        If $G_S$ is connected, but there exists a cycle in $G_S$, we can remove edges from S until it becomes a tree $G_S' = (V, S')$. Assume we have totally removed $k \geq 1$ edges. Then we have $|S| = n-1 = |V|-1 = |S'| = |S|-k$, which causes a contradiction. So this scenario can never happen.
        \item
        If $G_S$ is not connected. Then we can see it as $k$ connected component $G_i = (V_i, S_i)$, where $1 \leq i \leq k$. If all components are trees.  Denote $|V_i|=n_i$, $|S_i|=m_i$. So we have $n-1=\sum_{i=1}^n n_i-1 = \sum_{i=1}^k (m_i+1)-1 = \sum _{i=1}^k m_i+p-1 = m+p-1 = n+p-2$. So we have $p=1$, which leads to the contraction to $G_S$ is not connected. Hence, there exists a cycle, assume $G_1$ is a cycle. Using question 6.2, $d_1$ is not full rank. So $det(\hat{d_1})=0$. Therefore $det(\hat{d_S})=0$. That means $det(\hat{d_S}^T) det(\hat{d_S})=0$.
    \end{enumerate}
    From all above, we can conclude that $L_{1,1}$ is the number of the spanning trees of $G$. \#
     
    
    
    
\end{enumerate}




\section*{Problem 7}
\begin{enumerate}
    \item 
    \begin{lemma}
    Let $x \in \mathbb{R}^n$, $x=(x_1, \cdots,x_n)^T$. L is the Laplacian matrix of graph $G = (V,E)$. Then $x^TLx=\sum_{(v_i,v_j)\in E} (x_i-x_j)^2$.
    \end{lemma}
    \begin{proof}
    By definition of Laplacian $L=D-A$, we have
    \begin{align*}
        x^TLx &= x^TDx - x^T Ax \\
        &= (x_1, \cdots,x_n) \text{diag}(\text{deg}(v_1), \cdots, \text{deg}(v_n))(x_1, \cdots,x_n)^T\\
        &-(x_1, \cdots,x_n)A(x_1, \cdots,x_n)^T\\
        &=\sum_{i=1}^n \text{deg}(v_i)x_i^2 - (x_1, \cdots,x_n)A(x_1, \cdots,x_n)^T
    \end{align*}
    For each entries $a_{ij}$ in $A$, $a_{ij}=1 $ if there exists an edge $(v_i,v_j) \in E$. \newline So $(x_1, \cdots,x_n)A(x_1, \cdots,x_n)^T = \sum_{(v_i,v_j)\in E} 2x_i x_j$.
    $$x^TLx = \sum_{i=1}^n \text{deg}(v_i)x_i^2 - 2\sum_{(v_i,v_j)\in E} x_i x_j = \sum_{(v_i,v_j)\in E}(x_i-x_j)^2$$
    \end{proof}
    \begin{lemma}
    If $G$ is a connected graph, the algebraic multiplicity of eigenvalue 0 of its Laplacian is 1.
    \end{lemma}
    \begin{proof}
    Suppose the algebraic multiplicity of eigenvalue 0 is greater than 1. Then there exists two linearly independence eigenvectors $x_1,x_2$ with eigenvalue 0. 
    We have $L x_1=0$, $L x_2=0$.
    From Lemma 2, we have $\sum_{(i,j)\in E} (x_{1_i} -x_{1_j})^2 = x_1^T L x_1=0$, and $\sum_{(i,j)\in E} (x_{2_i} -x_{2_j})^2 = x_2^T L x_2=0$. So for $\forall (i,j)\in E$, $x_{1_i} =x_{1_j}$ and $x_{2_i} =x_{2_j}$. Because $G$ is connected, there is path between any pair of vertices. Therefore $x_{1_i} = c_1, x_{2_i} = c_2,$ for $1 \leq i \leq n$. That means $x_1,x_2$ are both in span\{\textbf{1}\}, which causes a contradiction to the independence of $x_1$ and $x_2$.
    \end{proof}
    
    Here we have a graph $G$ of $k$ connected components. Then the adjacency matrix can be block diagonal with $k$ blocks. Since the degree matrix is naturally a diagonal matrix, the Laplacian $L=D-A$ can also be block diagonal with $k$ blocks $L_1, \cdots,L_k$. By Lemma 3, each block matrix has eigenvalue 0 with multiplicity one. Then $det(\lambda I-L) = det(\lambda I -L_1)\cdots(\lambda I -L_k)$. Hence the algebraic multiplicity of eigenvalue 0 in $L$ is the sum of algebraic multiplicity of eigenvalue 0 in each $L_i$, that is equal to k. Therefore, the number of nonzero eigenvalues of $L$ is precisely $n-k$. \#
    
    
    
    \item
    For embedding $f \in \mathbb{R}^V$, so that $f(i)=x_i$, for $i \in V$. We want to minimize the energy $$\mathcal{E}=\sum_{e \in E} l_e^2 = \sum_{(i,j) \in E} (x_i - x_j)^2$$
    In the meantime we have the constraint that the moment of inertia is 1, that is,
    $$I=\sum_{i \in V}x_i^2  = 1$$
    Let $x=(x_1,\cdots,x_n)^T$. Then our constraint becomes $x^Tx=1$.
    Then for the energy function, 
    \begin{align*}
        \mathcal{E} &= \sum_{(i,j)\in E}(x_i-x_j)^2 \\
        &=\sum_{(i,j)\in E} (x_i^2+x_j^2-2x_ix_j)\\
        &=\sum_{(i,j) \in E}x_i^2 + \sum_{(i,j) \in E}x_j^2 - 2\sum_{(i,j) \in E}x_ix_j\\
        &=\sum_{i \in V} deg(x_i)x_i^2 + \sum_{(i,j) \in E,i<j}(-1)*x_ix_j + \sum_{(i,j) \in E,i>j}(-1)*x_ix_j\\
        &=x^TLx
    \end{align*}
    Here $L$ is the Laplacian, and it holds because
     \begin{equation*}
        l_{pq} = 
        \begin{cases}
        \text{deg}(v_p)& p=q\\
        -1& p < q \text{ and} (v_p,v_q) \in E\\
        -1& p > q \text{ and} (v_p,v_q) \in E\\
        0& o.w.
        \end{cases}
    \end{equation*}
    Then to ensure the center of masses is equal to 0. Introduce the centering matrix $H = I -\frac{1}{n}\textbf{1}\textbf{1}^T$. Because
    $\sum_{(i,j)\in E}(x_i-x_j)^2 = \sum_{(i,j) \in E}[(x_i-\bar{x})-(x_j-\bar{x})]^2$, so we can transfer the original question to:
    $$\underset{x}{\min}\quad x^THLHx$$
    $$s.t.\quad x^T Hx=1$$
    
    Let $y=Hx$. Because $H\textbf{1}=0$, $HH = H(I-\frac{1}{n}\textbf{1}\textbf{1}^T)=H$. Then $y^Ty = x^THHy=x^THx=1$. And $x^THLHx = y^TLy$.

    So we transfer our optimization problem to:
    $$\underset{y}{\min}\quad y^TLy$$
    $$s.t.\quad y^T y=1$$ 
    
    Because $L=d^Td$, where $d$ is the derivative of the graph, $y^TLy=y^Td^Tdy=(dy)^T(dy)=||dy||_2^2$.  Also $y^Ty=||y||_2^2$. Hence we transfer the optimization to:
    $$\underset{y}{\min}||dy||_2^2$$
    $$s.t.\quad ||y||_2^2=1$$ 
    This is a problem to solve $d$'s minimal singular value and its corresponding singular vectors. But now we notice that $\textbf{1}y = \textbf{1}Hx=0$, so we need $y$ to be vertical to \textbf{1}.
    Because $L$ has a eigenvalue 0, so $d$ has a singular value 0, which is the minimal singular value for $d$. However $\ker (L)=span(\textbf{1})$, so the singular vector corresponding to 0 is \textbf{1}, which is not vertical to \textbf{1}. Therefore, we cannot use the minimal singular value 0. We consider the singular values of $d$, by the definition of singular values, they are the square root of eigenvalues of $L$. By question 7.1 we know the  algebraic multiplicity of eigenvalue 0 of $L$ is 1. So the algebraic multiplicity of the singular value 0 of $d$ is 1. Therefore the solution is the minimal non-zero singular value of $d$, and $y$ is equal to its corresponding singular vectors. This is correct because the singular vectors are orthogonal.
    
    In a nutshell, we find the tight lower bound is the minimal non-zero singular value of $d$, and the embedding $f$ achieves this bound is to map $\{i\}$ to the singular vector corresponding to the minimal non-zero singular value of $d$. 
     

    
    
    





\end{enumerate}


\end{document} 
